\documentclass[a4paper]{mwart}

% PACKAGES
\usepackage[utf8]{inputenc}
% \usepackage{fontspec}
\usepackage[T1]{fontenc}
\usepackage{float}
\usepackage{graphicx}
\usepackage{listings}

\title{Gliwice360 \\ Aplikcja do zwiedzania Gliwic}
\author{Szymon Hankus}
\date{\today}
% Aby skasować datę, można wstawić \date{} (z pustymi parametrami)

\begin{document}
\maketitle
\tableofcontents

\section{Wstęp}
Ten dokument ma na celu przedstawienie problemu projektowego wraz z jego
rozwiązaniem. Projekt został przygotowany jako jeden z elementów potrzebnych do
zaliczenia przedmiotu ,,Fotografia cyfrowa'' prowadzonego przez doktora Piotra
Gawrona.

\subsection{Motywacja}
Celem projektu było oczywiście wykonanie estetycznych zdjęć, lecz miały być one
częścią jakiegoś głębszego pomysłu. Po ostatnich (moich pierwszych) wyborach
samorządowych zacząłem interesować się sprawami lokalnymi i doceniać wartość 
samorządu, a także wkładu, który każdy może w nim mieć. Mój pomysł -- aplikacja
z panoramami sferycznymi różnych miejsc w Gliwicach umiejscowionych na mapie --
jest więc wyrazem lokalnego patriotyzmu do mojego rodzinnego miasta.

\section{Pomysł}

\section{Napotkane problemy}

\section{Wnioski}


\end{document}
